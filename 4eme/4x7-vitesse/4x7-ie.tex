\documentclass[11pt]{article}
\usepackage{geometry,marginnote} % Pour passer au format A4
\geometry{hmargin=1cm, vmargin=1cm} % 

% Page et encodage
\usepackage[T1]{fontenc} % Use 8-bit encoding that has 256 glyphs
\usepackage[english,french]{babel} % Français et anglais
\usepackage[utf8]{inputenc} 

\usepackage{lmodern,numprint}
\setlength\parindent{0pt}

% Graphiques
\usepackage{graphicx,float,grffile,units}
\usepackage{tikz,pst-eucl,pst-plot,pstricks,pst-node,pstricks-add,pst-fun} 

% Maths et divers
\usepackage{amsmath,amsfonts,amssymb,amsthm,verbatim}
\usepackage{multicol,enumitem,url,eurosym,gensymb,tabularx}

\DeclareUnicodeCharacter{20AC}{\euro}



% Sections
\usepackage{sectsty} % Allows customizing section commands
\allsectionsfont{\centering \normalfont\scshape}

% Tête et pied de page
\usepackage{fancyhdr} \pagestyle{fancyplain} \fancyhead{} \fancyfoot{}

\renewcommand{\headrulewidth}{0pt} % Remove header underlines
\renewcommand{\footrulewidth}{0pt} % Remove footer underlines

\newcommand{\horrule}[1]{\rule{\linewidth}{#1}} % Create horizontal rule command with 1 argument of height

\newcommand{\Pointilles}[1][3]{%
  \multido{}{#1}{\makebox[\linewidth]{\dotfill}\\[\parskip]
}}

\newtheorem{Definition}{Définition}

\usepackage{siunitx}
\sisetup{
    detect-all,
    output-decimal-marker={,},
    group-minimum-digits = 3,
    group-separator={~},
    number-unit-separator={~},
    inter-unit-product={~}
}

\setlength{\columnseprule}{1pt}

\begin{document}

\textbf{Nom, Prénom :} \hspace{8cm} \textbf{Classe :} \hspace{3cm} \textbf{Date :}\\

\begin{center}
  \textit{Le courage est le juste milieu entre la peur et l'audace}  - \textbf{Aristote}
\end{center}

\subsection*{Cours}

\begin{multicols}{3}
  \begin{enumerate}
  \item[v=] \dotfill 
  \item[d=] \dotfill
  \item[t=] \dotfill 
  \end{enumerate}
\end{multicols}

\subsection*{Exercices}
\textbf{Merci de rédiger et d'écrire les calculs !}

\paragraph{\textbf{Ex1 : }} Yacine cours le 300m en 40s. Quelle est sa vitesse moyenne en m/s ?
\newline \Pointilles[5]

\paragraph{\textbf{Ex2 : }} Jasmine regarde son compteur sur son vélo. Il lui indique qu'elle a roulé pendant 1h30min à une vitesse moyenne de 32km/h. Quelle est la distance parcourue ?
\newline \Pointilles[5]

\paragraph{\textbf{Ex3 : }} Sur l'autoroute Mustapha roule à la vitesse constante de 130km/h. Combien de temps va-t-il mettre pour atteindre la prochaine sortie située à 25km ?
\newline \Pointilles[5]

\paragraph{\textbf{Ex4 : }} Rosa regarde le glacier du Mont-Blanc avancé à la vitesse de 8m par jour. De combien aura-t-il avancé en : 

\begin{multicols}{4}
  \begin{enumerate}
    \item[1.] 10 jours ?
    \item[2.] 365 jours ?
    \item[3.] 4 siècles ?
    \item[4.] 10 secondes ? 
  \end{enumerate}
\end{multicols}

\Pointilles[10]

\newpage

\paragraph{\textbf{Ex5 : }} Afin de voir le concert de JuL, Yanis prend le TGV au départ de Villefranche-sur-Saône à 8h45 et arrive à Marseille à 10h24. Le deux villes sont distantes de 350km. Quelle est la vitesse moyenne du TGV ?

\Pointilles[14]

\paragraph{\textbf{Ex6 : }} Rania mets 12 min pour aller chercher des fleurs pour sa maman chez le fleuriste qui se situe à 4,2km de chez elle. En conservant cette allure, quelle distance est parcourue en 2h36min ?

\Pointilles[14]

\paragraph{\textbf{Ex7 : }} Assia est câblé avec la fibre et télécharge en moyenne à 30Mo par seconde. Elle souhaite faire la mise à jour de 4 Go de son jeu favori sur Steam. Elle commence la mise à jour à 18h15. A quelle heure la mise à jour sera-t-elle finie ? (rappel 1 Go = 1 000 Mo)

\Pointilles[14]

\newpage

\textbf{Nom, Prénom :} \hspace{8cm} \textbf{Classe :} \hspace{3cm} \textbf{Date :}\\

\begin{center}
  \textit{Le courage est le juste milieu entre la peur et l'audace}  - \textbf{Aristote}
\end{center}

\subsection*{Cours}

\begin{multicols}{3}
  \begin{enumerate}
  \item[v=] \dotfill 
  \item[d=] \dotfill
  \item[t=] \dotfill 
  \end{enumerate}
\end{multicols}

\subsection*{Exercices}
\textbf{Merci de rédiger et d'écrire les calculs !}

\paragraph{\textbf{Ex1 : }} Yacine met 30s pour courir 250m. Quelle est sa vitesse moyenne en m/s ?
\newline \Pointilles[5]

\paragraph{\textbf{Ex2 : }} Jasmine regarde son compteur sur son vélo. Il lui indique qu'elle a roulé pendant 1h30min à une vitesse moyenne de 36km/h. Quelle est la distance parcourue ?
\newline \Pointilles[5]

\paragraph{\textbf{Ex3 : }} Sur l'autoroute Mustapha roule à la vitesse constante de 130km/h. Combien de temps va-t-il mettre pour atteindre la prochaine sortie située à 16km ?
\newline \Pointilles[5]

\paragraph{\textbf{Ex4 : }} Rosa regarde le glacier du Mont-Blanc avancé à la vitesse de 6m par jour. De combien aura-t-il avancé en : 

\begin{multicols}{4}
  \begin{enumerate}
    \item[1.] 10 jours ?
    \item[2.] 365 jours ?
    \item[3.] 6 siècles ?
    \item[4.] 30 secondes ? 
  \end{enumerate}
\end{multicols}

\Pointilles[10]

\newpage

\paragraph{\textbf{Ex5 : }} Afin de voir le concert de JuL, Yanis prend le TGV au départ de Villefranche-sur-Saône à 7h15 et arrive à Marseille à 9h34. Le deux villes sont distantes de 350km. Quelle est la vitesse moyenne du TGV ?

\Pointilles[14]

\paragraph{\textbf{Ex6 : }} Rania mets 16 min pour aller chercher des fleurs pour sa maman chez le fleuriste qui se situe à 4,8km de chez elle. En conservant cette allure, quelle distance est parcourue en 2h36min ?

\Pointilles[14]

\paragraph{\textbf{Ex7 : }} Assia est câblé avec la fibre et télécharge en moyenne à 20Mo par seconde. Elle souhaite faire la mise à jour de 3 Go de son jeu favori sur Steam. Elle commence la mise à jour à 17h05. À quelle heure la mise à jour sera-t-elle finie ? (rappel 1 Go = 1 000 Mo)

\Pointilles[14]

\end{document}

