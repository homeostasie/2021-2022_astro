\documentclass[11pt]{article}
\usepackage{geometry,marginnote} % Pour passer au format A4
\geometry{hmargin=1cm, vmargin=1cm} % 

% Page et encodage
\usepackage[T1]{fontenc} % Use 8-bit encoding that has 256 glyphs
\usepackage[english,french]{babel} % Français et anglais
\usepackage[utf8]{inputenc} 

\usepackage{lmodern,numprint}
\setlength\parindent{0pt}

% Graphiques
\usepackage{graphicx,float,grffile,units}
\usepackage{tikz,pst-eucl,pst-plot,pstricks,pst-node,pstricks-add,pst-fun} 

% Maths et divers
\usepackage{amsmath,amsfonts,amssymb,amsthm,verbatim}
\usepackage{multicol,enumitem,url,eurosym,gensymb,tabularx}

\DeclareUnicodeCharacter{20AC}{\euro}



% Sections
\usepackage{sectsty} % Allows customizing section commands
\allsectionsfont{\centering \normalfont\scshape}

% Tête et pied de page
\usepackage{fancyhdr} \pagestyle{fancyplain} \fancyhead{} \fancyfoot{}

\renewcommand{\headrulewidth}{0pt} % Remove header underlines
\renewcommand{\footrulewidth}{0pt} % Remove footer underlines

\newcommand{\horrule}[1]{\rule{\linewidth}{#1}} % Create horizontal rule command with 1 argument of height

\newcommand{\Pointilles}[1][3]{%
  \multido{}{#1}{\makebox[\linewidth]{\dotfill}\\[\parskip]
}}

\newtheorem{Definition}{Définition}

\usepackage{siunitx}
\sisetup{
    detect-all,
    output-decimal-marker={,},
    group-minimum-digits = 3,
    group-separator={~},
    number-unit-separator={~},
    inter-unit-product={~}
}

\setlength{\columnseprule}{1pt}

\begin{document}

\textbf{Nom, Prénom :} \hspace{8cm} \textbf{Classe :} \hspace{3cm} \textbf{Date :}\\

\begin{center}
  \textit{If you do the work, you get rewarded. There are no shortcuts in life}  - \textbf{Michael Jordan}
\end{center}

\subsubsection*{Ex 1 : Représenter et Calculer}
\textit{Écrire la puissance à l'aide de l'opération $\times$ puis calculer.}

\begin{multicols}{3}
  \begin{itemize}
    \item[a.] $7^2 + 1 $ \\
              =  \dotfill \\
              =  \dotfill            
    \item[b.] $3 \times 10^5 $ \\
              =  \dotfill \\
              =  \dotfill      
    \item[c.] $5^3 \times (4^2)^3$ \\
              =  \dotfill \\
              =  \dotfill      
    \item[d.] $(-2)^{4}$ \\
              =  \dotfill \\
              =  \dotfill      
    \item[e.] $ 10 + 2^1 \times 3^2 $ \\
              =  \dotfill \\
              =  \dotfill      
    \item[f.] $4^{-2}$ \\
              =  \dotfill \\
              =  \dotfill      
  \end{itemize}

\end{multicols}

\subsubsection*{Ex 2 : Calculer}

\begin{multicols}{2}
  \begin{itemize}
    \item[g.] $\dfrac{0,15 \times 10^{-4} \times 2,7 \times 10^{2}}{500 \times (10^5)^2} =  \dotfill $
    \item[h.] $\dfrac{60 \times 10^{-11} \times 0,18 \times 10^{-3}}{4,8 \times (10^{-7})^5} =  \dotfill $ 
  \end{itemize}
\end{multicols}


\subsubsection*{Ex 3 : Les règles de calculs}

\textit{Écrire le résultat sous la forme d'une puissance en utilisant les règles.}


\begin{multicols}{4}
  \begin{enumerate}
  \item[i.] $6^{7}  \times  6^{8}  =  \dotfill$
  \item[j.] $\dfrac{10^{12}}{10^{12}} = \dotfill$
  \item[k.] $5^{6} \times 5^{5} = \dotfill$
  \item[l.] $\dfrac{11^{11}}{11^{6}} = \dotfill$
  \item[m.] $11^{9} \times 7^{9} = \dotfill$
  \item[n.] $5^{6} \times 4^{6} = \dotfill$
  \item[o.] $(12^{10})^{8} = \dotfill$
  \item[p.] $(10^{10})^{7} = \dotfill$
  \end{enumerate}
\end{multicols}

\subsubsection*{Ex 4 : Démontrer}

\textit{Écrire la démonstration pour $(2^4)^3$} \\
\Pointilles[5]

\subsubsection*{Ex 5 : Écrire sous forme scientifique.}

\begin{multicols}{2}

  \begin{enumerate}
  \item[q.] $\SI{70000}{} = \dotfill$
  \item[r.] $\SI{1420000000}{} = \dotfill$
  \item[s.] $\SI{-2340000}{} = \dotfill$
  \item[t.] $\SI{-145000}{} = \dotfill$
  \item[u.] $\SI{0.004}{} = \dotfill$
  \item[v.] $\SI{0.000000567}{} = \dotfill$
  \item[w.] $\SI{-0.45}{} = \dotfill$
  \item[x.] $\SI{-0.00408}{} = \dotfill$
  \end{enumerate}
\end{multicols}

\subsubsection*{Ex 6 : L'objectif de la notation scientifique}
\Pointilles[2]

\subsubsection*{Ex Bonus : Écrire un calcul que vous savez faire mais pas votre calculatrice}
\Pointilles[2]


\newpage

\subsubsection*{Problème 1 - Masse d'un trou noir}

La masse de la terre est $M_T = 5,9 \times 10^24 kg$. La masse d'un trou noir est $\SI{20000000}{}$ fois plus lourde. 

\textbf{Quelle est la masse d'un trou noir ?}

\Pointilles[4]

\subsubsection*{Problème 2 - Vitesse de la lumière}

La lumière se déplace à la vitesse de $3 \times 10^8$ m/s. 

\textbf{Quelle distance parcourt-elle en cinq jours ?}

\Pointilles[4]

\subsubsection*{Problème 3 - Étoiles Vs Grain de sables.}

Il y a $2^{30}$ galaxies dans notre univers. Chaque galaxies contient $3^{31}$ étoiles.  \\
On estime le volume de sable sur Terre à $1\,200 \text{ milliards de } m^3$. Chaque $m^3$ contient environ $510 \text{milliards}$ de grains de sable. 

\textbf{Fairouz affirme qu'il y a plus de grains de sable sur Terre que d'étoiles dans l'univers ? A-t-elle raison ?}

\Pointilles[4]

\subsubsection*{Problème 4 - Casa de Papel}

\textit{\og El Professeur \fg{} } vient de dérober 12 millions d’euros. \\
Les billets de banque ont une épaisseur de $60 \times 10^{-6} m$. (On dit 60 micromètres)

\textbf{Quelle hauteur atteindrait une pile de billets de banque de 50 \euro{} représentant cette somme ?}

\Pointilles[5]

\subsubsection*{Problème 5 - $CO_2$  - \textit{(/4)}}

Osvaldo fait une expérience de Chimie. Une molécule de dioxyde de carbone est composée d'un atome de carbone et de deux atomes d'oxygène. La masse d'un atome de carbone est $2 \times 10^{-26}kg$ et la masse d'un atome d'oxygène est $1.8 \times 10^{-26}kg$. 

\textbf{Combien trouve-t-on de molécule de dioxyde carbone dans 2kg ?}

\Pointilles[5]

\newpage


\textbf{Nom, Prénom :} \hspace{8cm} \textbf{Classe :} \hspace{3cm} \textbf{Date :}\\

\begin{center}
  \textit{If you do the work, you get rewarded. There are no shortcuts in life}  - \textbf{Michael Jordan}
\end{center}

\subsubsection*{Ex 1 : Représenter et Calculer}
\textit{Écrire la puissance à l'aide de l'opération $\times$ puis calculer.}

\begin{multicols}{3}
  \begin{itemize}
    \item[a.] $5^3 + 1 $ \\
              =  \dotfill \\
              =  \dotfill            
    \item[b.] $4 \times 10^4 $ \\
              =  \dotfill \\
              =  \dotfill      
    \item[c.] $3^5 \times (3^3)^2$ \\
              =  \dotfill \\
              =  \dotfill      
    \item[d.] $(-5)^{4}$ \\
              =  \dotfill \\
              =  \dotfill      
    \item[e.] $ 100 + 5^1 \times 2^3 $ \\
              =  \dotfill \\
              =  \dotfill      
    \item[f.] $8^{-3}$ \\
              =  \dotfill \\
              =  \dotfill      
  \end{itemize}

\end{multicols}

\subsubsection*{Ex 2 : Calculer}

\begin{multicols}{2}
  \begin{itemize}
    \item[g.] $\dfrac{0,25 \times 10^{-3} \times 2,6 \times 10^{3}}{400 \times (10^4)^2} =  \dotfill $
    \item[h.] $\dfrac{80 \times 10^{-12} \times 0,14 \times 10^{-2}}{4,7 \times (10^{-6})^5} =  \dotfill $ 
  \end{itemize}
\end{multicols}


\subsubsection*{Ex 3 : Les règles de calculs}

\textit{Écrire le résultat sous la forme d'une puissance en utilisant les règles.}


\begin{multicols}{4}
  \begin{enumerate}
  \item[i.] $8^{6}  \times  8^{6}  =  \dotfill$
  \item[j.] $\dfrac{10^{15}}{10^{10}} = \dotfill$
  \item[k.] $8^{6} \times 8^{8} = \dotfill$
  \item[l.] $\dfrac{13^{11}}{13^{5}} = \dotfill$
  \item[m.] $11^{6} \times 8^{6} = \dotfill$
  \item[n.] $5^{8} \times 3^{8} = \dotfill$
  \item[o.] $(12^{11})^{7} = \dotfill$
  \item[p.] $(10^{11})^{6} = \dotfill$
  \end{enumerate}
\end{multicols}

\subsubsection*{Ex 4 : Démontrer}

\textit{Écrire la démonstration pour $(5^3)^4$} \\
\Pointilles[5]

\subsubsection*{Ex 5 : Écrire sous forme scientifique.}

\begin{multicols}{2}

  \begin{enumerate}
  \item[q.] $\SI{4000}{} = \dotfill$
  \item[r.] $\SI{143000000}{} = \dotfill$
  \item[s.] $\SI{-73400000}{} = \dotfill$
  \item[t.] $\SI{-6450000}{} = \dotfill$
  \item[u.] $\SI{0.03}{} = \dotfill$
  \item[v.] $\SI{0.00000000967}{} = \dotfill$
  \item[w.] $\SI{-0.25}{} = \dotfill$
  \item[x.] $\SI{-0.00308}{} = \dotfill$
  \end{enumerate}
\end{multicols}

\subsubsection*{Ex 6 : L'objectif de la notation scientifique}
\Pointilles[2]

\subsubsection*{Ex Bonus : Écrire un calcul que vous savez faire mais pas votre calculatrice}
\Pointilles[2]


\newpage

\subsubsection*{Problème 1 - Masse d'un trou noir}

La masse de la terre est $M_T = 5,9 \times 10^27 g$. La masse d'un trou noir est $\SI{20000000}{}$ fois plus lourde. 

\textbf{Quelle est la masse d'un trou noir ?}

\Pointilles[4]

\subsubsection*{Problème 2 - Vitesse de la lumière}

La lumière se déplace à la vitesse de $3 \times 10^8$ m/s. 

\textbf{Quelle distance parcourt-elle en quatre jours ?}

\Pointilles[4]

\subsubsection*{Problème 3 - Étoiles Vs Grain de sables.}

Il y a $2^{30}$ galaxies dans notre univers. Chaque galaxies contient $3^{31}$ étoiles.  \\
On estime le volume de sable sur Terre à $1\,200 \text{ milliards de } m^3$. Chaque $m^3$ contient environ $580 \text{milliards}$ de grains de sable. 

\textbf{Fairouz affirme qu'il y a plus de grains de sable sur Terre que d'étoiles dans l'univers ? A-t-elle raison ?}

\Pointilles[4]

\subsubsection*{Problème 4 - Casa de Papel}

\textit{\og El Professeur \fg{} } vient de dérober 15 millions d’euros. \\
Les billets de banque ont une épaisseur de $80 \times 10^{-6} m$. (On dit 60 micromètres)

\textbf{Quelle hauteur atteindrait une pile de billets de banque de 20 \euro{} représentant cette somme ?}

\Pointilles[5]

\subsubsection*{Problème 5 - $CO_2$  - \textit{(/4)}}

Osvaldo fait une expérience de Chimie. Une molécule de dioxyde de carbone est composée d'un atome de carbone et de deux atomes d'oxygène. La masse d'un atome de carbone est $2 \times 10^{-26}kg$ et la masse d'un atome d'oxygène est $1.8 \times 10^{-26}kg$. 

\textbf{Combien trouve-t-on de molécule de dioxyde carbone dans kg ?}

\Pointilles[5]
\end{document}
