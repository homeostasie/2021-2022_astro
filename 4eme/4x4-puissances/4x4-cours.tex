\input{../doc-class-cours.tex}

\begin{document}

\horrule{2px}
\section*{Chapitre 4 - Puissances}
\horrule{2px}

\section*{Les connaissances}

\begin{multicols}{2}

  \begin{enumerate}
    \item[1.] Apprendre la nouvelle opération.
    \item[2.] Apprendre les règles de calculs. 
    \item[3.] Apprendre la notation scientifique.
    \item[4.] Savoir résoudre des problèmes. 
  \end{enumerate}

\end{multicols}

\section*{1 - L'opération puissance}


\subsection*{Une nouvelle opération}

\textbf{L'opération puissance est une nouvelle opération}. On multiplie un nombre par lui-même un certain nombre de fois. En ce point, elle ressemble à la multiplication. \textit{On lit deux puissance quatre}. 

\begin{flalign*}
  2 + 2 + 2 + 2 &= 2 \times 4 \\
  2 \times 2 \times 2 \times 2 &= 2^4 \\
\end{flalign*}

Quelques exemples : 

\begin{flalign*}
  5^6  &= 5 \times 5 \times 5 \times 5 \times 5 \times 5 \\
  (-3)^4  &= -3 \times -3 \times -3 \times -3 
\end{flalign*}
  
\textit{Il va falloir être prudent avec les signes $-$.}

\subsection*{Les puissances de 2 et de 10}

 Les puissances de 2 sont utilisées à la base de l'informatique pour stocker des nombres en mémoire dans le système de numération binaire. Les puissances de 10 sont utilisées pour écrire des nombres très grands.

\begin{multicols}{2}

  \underline{Liste des premières puissances de 2 :} 

  \begin{flalign*}
   2^1 &= 2 \\
   2^2 &= 2 \times 2 = 4 \\
   2^3 &= 2 \times 2 \times 2 = 8 \\
   2^4 &= 2 \times 2 \times 2 \times 2 = 16 \\
   2^5 &= 2 \ldots 2 = 32 \\
   2^6 &= 2 \ldots 2 = 64 \\
   2^7 &= 2 \ldots 2 = 128 \\
   2^8 &= 2 \ldots 2 = 256 \\
   2^9 &= 2 \ldots 2 = 512 \\
   2^{10} &= 2 \ldots 2 = 1024 \\
  \end{flalign*}

  \columnbreak

  \underline{Liste des premières puissances de 10 :} 

  \begin{flalign*}
  10^0 &= 1 \text{ // Par définition.} \\
  10^1 &= 10 \text{ // On multiplie par dix.} \\
  10^2 &= 100 \text{ //  On multiplie par cent.} \\
  10^3 &= 1 \, 000 \text{ // On multiplie par mille.} \\
  10^4 &= 10 \, 000 \text{ // On multiplie par dix mille.} \\
  10^5 &= 100 \, 000 \text{ // On multiplie par cent mille.} \\
  10^6 &= 1 \, 000\, 000 \text{ // On multiplie par un million.} \\
  10^7 &= 10 \, 000\, 000 \\
  10^8 &= 100 \, 000\, 000  \\
  10^9 &= 1 \, 000\, 000 \, 000 \text{ // On multiplie par un milliard.}\\
  10^{12} &=1 \, 000 \, 000 \, 000 \, 000 \\
\end{flalign*}

\end{multicols}

\section*{2 - Les règles de calculs}

Il existe \textbf{des règles} pour simplifier \textit{à la main} les puissances dans les calculs. Le but n'est pas de calculer mais de proposer une écriture plus simple.


\begin{multicols}{2}
  \begin{itemize}[label={$\bullet$}]
    \item \textbf{Règle 1 : } $ 2^{10} \times 2^{12} = 2^{10 + 12} = 2^{22} $
    \item \textbf{Règle 2 : } $ \dfrac{14^{25}}{14^{10}} = 14^{25 - 10} = 14^{15} $  
    \item \textbf{Règle 3 : } $ (10^4)^9 = 10^{4 \times 9} = 10^{36} $
    \item \textbf{Règle 4 : } $ 5^{19} \times 7^{19} = (5 \times 7)^{19} = 35^{19} $
  \end{itemize}
\end{multicols}
  
\section*{3. - Les puissances négatives}

Retournement de situation, il n'y a pas une nouvelle opération à découvrir dans ce chapitre mais deux. 

L'opération puissance fonctionne aussi avec des \textbf{exposants négatifs}. On multiplie alors un certain nombre de fois par l'inverse du nombre de départ. 

\begin{flalign*}
  2^{-4} &= \dfrac{1}{2} \times \dfrac{1}{2} \times \dfrac{1}{2} \times \dfrac{1}{2} = \dfrac{1}{ 2 \times 2 \times 2 \times 2} = \dfrac{1}{2^4}
\end{flalign*}

\underline{Liste des premières puissances de 10 négatives:} 

\begin{minipage}[t]{0.65\textwidth}
  \begin{flalign*}
    10^0 &= 1 \text{  // Par définition } \\
    10^{-1} &= 0,1 \text{  // On multiplie par un dixième. } \\
    10^{-2} &= 0,01 \text{  // On multiplie par centième. } \\
    10^{-3} &= 0,001 \text{  // On multiplie par un millième. } \\
  \end{flalign*}
\end{minipage}
\begin{minipage}[t]{0.3\textwidth}
  \begin{flalign*}
    10^{-4} &= 0,000 \, 1 \\
    10^{-5} &= 0,000 \, 01\\
    10^{-6} &= 0,000 \, 001\\
    10^{-9} &= 0,000 \, 000 \, 000 \, 1\\
  \end{flalign*}
\end{minipage}

\section*{4 - La notation scientifique}

\subsection*{Les grands nombres}

\begin{multicols}{2}

  L'opération puissance permet d'obtenir rapidement de très grands nombres.

  $$2^{64} = 18 \, 446 \, 744 \, 073 \, 709 \, 551 \, 616$$

  \paragraph{Problème :}

  \begin{itemize}
  \item Ce nombre est difficile à lire.
  \item Ce nombre est long à représenter.
  \item Ce nombre est source d'erreur à écrire.
  \end{itemize}

  Les calculatrices présentent les résultats autrement : \textbf{la forme scientifique.}

  $$2^{64} = 1,844 \, 674 \, 407 \times 10^{19}$$

  \columnbreak

  \underline{Pour écrire un nombre sous forme scientifique :} \\
  \begin{itemize}
  \item on place la virgule après le premier chiffre puis on multiplie par une puissance de 10. \\
  \end{itemize}

  La puissance de 10 s'appelle \textbf{l'ordre de grandeur.}\\
  \textit{On le lit : un virgule quatre-vingt-quatre fois dix puissance dix-neuf.}

  \paragraph{Conclusion :}

  \begin{itemize}
  \item La représentation de ce nombre tient à l'écran.
  \item Ce nombre est facile à lire et à écrire.
  \item On comprend tout de suite l'ordre de grandeur de ce nombre.
  \end{itemize}

    \textbf{La notation scientifique} utilise les puissances de dix. Pour connaître l'ordre de grandeur, on compte le nombre de chiffre entre le premier chiffre et le chiffre des unités. \\ 
\end{multicols}

\newpage

\subsection*{Les petits nombres}

\begin{multicols}{2}

  On rencontre les mêmes problèmes avec les petits nombres.

  $$2^{-10} = 0, 000 \, 976\, 562\, 5$$

  Pour les mêmes raisons, les calculatrices vont aussi les représenter sous \textbf{forme scientifique}.

  $$2^{-10} = 9, 7654 \, 625 \times 10^{-4}$$

  \textit{On le lit : neuf virgule soixante seize fois dix puissance moins quatre.} \\
  \textbf{La notation scientifique} utilise les puissances de dix négatives pour représenter les petits nombres. Afin de connaître l'ordre de grandeur, on compte le nombre de 0 en partant du premier. \\

\end{multicols}
 

\section*{5 - La science et le numérique}

Très rapidement, ce chapitre permet d'atteindre et de comprendre les limites de notre calculatrice. \\

\subsection*{Les pièges à la calculatrice}

  \begin{itemize}[label={$\bullet$}]
  \item $10^{12} + 1 = 10^{12} $. La calculatrice calcule \textbf{faux} mais affiche \textbf{correctement}. La preuve étant que $10^{12} + 1 - 10^{12} = 1$ à la calculatrice.
  \item $10^{14} + 1 = 10^{14} $. La calculatrice calcule \textbf{faux} et affiche \textbf{faux}. La preuve étant que $10^{14} + 1 - 10^{14} = 0$ ce qui est faux... On peut avoir le bon résultat en changeant l'ordre des calculs : $10^{14} - 10^{14} +1 = 1$.
  \item $\dfrac{10^{100}}{10^{100}} = \text{ERREUR !} $. Notre calculatrice ne sait pas calculer avec des nombres aussi grand et nous le dit... même si le résultat est petit..
  \end{itemize}

\subsubsection*{En quête du sens physique : le très grand}

\begin{multicols}{2}

  \begin{figure}[H]
        \centering
        \includegraphics[width=0.8\linewidth]{4x4-puissances/terre.png}
  \end{figure}

  \begin{flalign*}
        10^{16} - 10^{16} + 1,6 &= 1,6 \\
        10^{16} + 1,2 - 10^{16} &= 0 \text{\textbf{ !!!}}
  \end{flalign*}

  Pour nos calculatrices : $10^{16} + 1,6 = 10^{16}$. \\
  Scientifiquement, l'\textbf{approximation} est acceptable. \\
  La distance Terre-Soleil est la même que l'on mesure de notre tête ou de nos pieds.

\end{multicols}

\subsubsection*{En quête du sens physique : le très petit}

\begin{multicols}{2}

  \begin{figure}[H]
    \centering
    \includegraphics[width=0.8\linewidth]{4x4-puissances/papier.png}
  \end{figure}

  \begin{flalign*}
    12,5 - 12,5 + 10^{-16} &= 10^{-16} \\
    12,5 + 10^{-16} - 12,5 &= 0 \text{\textbf{ !!!}}
  \end{flalign*}

  Pour nos calculatrices : $10^{-16} + 1,6 = 1,6$.\\
  Scientifiquement, l'\textbf{approximation} est acceptable.\\
  Je mesure la même taille si je me mesure du sol ou si je marche sur une feuille de papier.
\end{multicols}

\end{document}
