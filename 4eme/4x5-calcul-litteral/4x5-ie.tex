\input{../doc-class-cours.tex}

\begin{document}

\section*{Programme de l’éval : Calcul Littéral}

\textbf{Date : } \hspace{4cm} \textit{Calculatrice autorisée}

\subsection*{Cours - (2pts)}


\begin{itemize}[label={$\bullet$}]
  \item Sens du signe = : Ce qui est à gauche = Ce qui est à droite. 
  \item $x ?$ : Le nombre x est inconnu et solution de l’équation. 
\end{itemize}

\subsection*{Exercices Types}

\subsubsection*{Ex1. Calculer - (3pts)}

On pose $x = 12$ et $b = 2,5$

\begin{itemize}[label={$\bullet$}]
  \item  $x + 2 - b = 11,5$
  \item  $5 \times x + 14 = 74$
  \item  $(2x - 4b)^4 = \SI{38416}{}$
\end{itemize}


\subsubsection*{Ex2. Réduire - (3pts)}

\begin{itemize}[label={$\bullet$}]
  \item $2x + 3x = 5x$
  \item $4x - x + 10 = 3x +10$
\end{itemize}

\subsubsection*{Ex3. Démontrer - (1pts)}
(une question)

Démontrer la réduction de (par exemple) 2x + 3x.

\begin{flalign*}
  2x + 3x &= x+x + x+x+x \\
          &= 5x
\end{flalign*}


\subsection*{Équations 4è (11pts)}

Résoudre. Mettre les étapes de calcul.

\begin{itemize}[label={$\bullet$}]
  \item $x + 12 = 20$
  \item $x - 4 = -13$
  \item $4x = 25$
  \item $5 = -2x$
  \item $2x + 3 = 5$
  \item $4 + 5x = 2 $
\end{itemize}

\end{document}