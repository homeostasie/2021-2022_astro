\documentclass[11pt]{article}
\usepackage{geometry,marginnote} % Pour passer au format A4
\geometry{hmargin=1cm, vmargin=1cm} % 

% Page et encodage
\usepackage[T1]{fontenc} % Use 8-bit encoding that has 256 glyphs
\usepackage[english,french]{babel} % Français et anglais
\usepackage[utf8]{inputenc} 

\usepackage{lmodern,numprint}
\setlength\parindent{0pt}

% Graphiques
\usepackage{graphicx,float,grffile,units}
\usepackage{tikz,pst-eucl,pst-plot,pstricks,pst-node,pstricks-add,pst-fun} 

% Maths et divers
\usepackage{amsmath,amsfonts,amssymb,amsthm,verbatim}
\usepackage{multicol,enumitem,url,eurosym,gensymb,tabularx}

\DeclareUnicodeCharacter{20AC}{\euro}



% Sections
\usepackage{sectsty} % Allows customizing section commands
\allsectionsfont{\centering \normalfont\scshape}

% Tête et pied de page
\usepackage{fancyhdr} \pagestyle{fancyplain} \fancyhead{} \fancyfoot{}

\renewcommand{\headrulewidth}{0pt} % Remove header underlines
\renewcommand{\footrulewidth}{0pt} % Remove footer underlines

\newcommand{\horrule}[1]{\rule{\linewidth}{#1}} % Create horizontal rule command with 1 argument of height

\newcommand{\Pointilles}[1][3]{%
  \multido{}{#1}{\makebox[\linewidth]{\dotfill}\\[\parskip]
}}

\newtheorem{Definition}{Définition}

\usepackage{siunitx}
\sisetup{
    detect-all,
    output-decimal-marker={,},
    group-minimum-digits = 3,
    group-separator={~},
    number-unit-separator={~},
    inter-unit-product={~}
}

\setlength{\columnseprule}{1pt}

\begin{document}

\section*{Programme de l’éval : Calcul Littéral}

\textbf{Date : } \hspace{4cm} \textit{Calculatrice autorisée}

\subsection*{Cours - (2pts)}


\begin{itemize}[label={$\bullet$}]
  \item Sens du signe = : Ce qui est à gauche = Ce qui est à droite. 
  \item $x ?$ : Le nombre x est inconnu et solution de l’équation. 
\end{itemize}

\subsection*{Exercices Types}

\subsubsection*{Ex1. Calculer - (3pts)}

On pose $x = 12$ et $b = 2,5$

\begin{itemize}[label={$\bullet$}]
  \item  $x + 2 – b = 11,5$
  \item  $5 \times x + 14 = 74$
  \item  $(2x – 4b)^4 = \SI{38416}{}$
\end{itemize}


\subsubsection*{Ex2. Réduire - (3pts)}

\begin{itemize}[label={$\bullet$}]
  \item $2x + 3x = 5x$
  \item $4x – x + 10 = 3x +10$
\end{itemize}

\subsubsection*{Ex3. Démontrer - (1pts)}
(une question)

Démontrer la réduction de (par exemple) 2x + 3x.

\begin{flalign*}
  2x + 3x &= x+x + x+x+x \\
          &= 5x
\end{flalign*}


\subsection*{Équations 4è (11pts)}

Résoudre. Mettre les étapes de calcul.

\begin{itemize}[label={$\bullet$}]
  \item $x + 12 = 20$
  \item $x – 4 = -13$
  \item $4x = 25$
  \item $5 = -2x$
  \item $2x + 3 = 5$
  \item $4 + 5x = 2 $
\end{itemize}

\end{document}