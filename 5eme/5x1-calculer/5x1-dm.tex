\documentclass[11pt]{article}
\usepackage{geometry,marginnote} % Pour passer au format A4
\geometry{hmargin=1cm, vmargin=1cm} % 

% Page et encodage
\usepackage[T1]{fontenc} % Use 8-bit encoding that has 256 glyphs
\usepackage[english,french]{babel} % Français et anglais
\usepackage[utf8]{inputenc} 

\usepackage{lmodern,numprint}
\setlength\parindent{0pt}

% Graphiques
\usepackage{graphicx,float,grffile}
\usepackage{pst-eucl, pst-plot,units} 

% Maths et divers
\usepackage{amsmath,amsfonts,amssymb,amsthm,verbatim}
\usepackage{multicol,enumitem,url,eurosym,gensymb}
\DeclareUnicodeCharacter{20AC}{\euro}

% Sections
\usepackage{sectsty} % Allows customizing section commands
\allsectionsfont{\centering \normalfont\scshape}

% Tête et pied de page

\usepackage{fancyhdr} \pagestyle{fancyplain} \fancyhead{} \fancyfoot{}

\renewcommand{\headrulewidth}{0pt} % Remove header underlines
\renewcommand{\footrulewidth}{0pt} % Remove footer underlines

\newcommand{\horrule}[1]{\rule{\linewidth}{#1}} % Create horizontal rule command with 1 argument of height

%----------------------------------------------------------------------------------------
%   Début du document
%----------------------------------------------------------------------------------------

\begin{document}

\setlength{\columnseprule}{1pt}

\begin{multicols}{2}
    \subsection*{ex1}

    \textbf{Recopier proprement les nombres suivants :}
    \nombre{42} ; \nombre{541} ; \nombre{7160} ; \nombre{989120} ; \nombre{15,42}.


    \subsection*{ex2}

    \textbf{Recopier proprement le calcul suivant :}
    \begin{align*}
        B &= 3 \times 2 \times 5 + 10 - 6 \div 3  \\
        B &= 6 \times 5 + 10 - 2\\
        B &= 30 + 10 - 2 \\
        B &= 40 - 2\\
        B &= 38\\  
    \end{align*}
\end{multicols}

\vspace{1cm}

\begin{multicols}{2}
    \subsection*{ex1}
    
    \textbf{Recopier proprement les nombres suivants :}
    \nombre{42} ; \nombre{541} ; \nombre{7160} ; \nombre{989120} ; \nombre{15,42}.
        
    \subsection*{ex2}
    
    \textbf{Recopier proprement le calcul suivant :}
    \begin{align*}
        B &= 3 \times 2 \times 5 + 10 - 6 \div 3  \\
        B &= 6 \times 5 + 10 - 2\\
        B &= 30 + 10 - 2 \\
        B &= 40 - 2\\
        B &= 38\\  
    \end{align*}
\end{multicols}

\vspace{1cm}

\begin{multicols}{2}
    \subsection*{ex1}
        
    \textbf{Recopier proprement les nombres suivants :}
    \nombre{42} ; \nombre{541} ; \nombre{7160} ; \nombre{989120} ; \nombre{15,42}.
        
        
    \subsection*{ex2}
        
    \textbf{Recopier proprement le calcul suivant :}
    \begin{align*}
        B &= 3 \times 2 \times 5 + 10 - 6 \div 3  \\
        B &= 6 \times 5 + 10 - 2\\
        B &= 30 + 10 - 2 \\
        B &= 40 - 2\\
        B &= 38\\  
    \end{align*}
\end{multicols}

\vspace{1cm}

\begin{multicols}{2}
    \subsection*{ex1}
        
    \textbf{Recopier proprement les nombres suivants :}
    \nombre{42} ; \nombre{541} ; \nombre{7160} ; \nombre{989120} ; \nombre{15,42}.
        
        
    \subsection*{ex2}
        
    \textbf{Recopier proprement le calcul suivant :}
    \begin{align*}
        B &= 3 \times 2 \times 5 + 10 - 6 \div 3  \\
        B &= 6 \times 5 + 10 - 2\\
        B &= 30 + 10 - 2 \\
        B &= 40 - 2\\
        B &= 38\\  
    \end{align*}
\end{multicols}

\end{document}
