\input{../doc-class-cours.tex}

\begin{document}

\subsection*{ex1 - Recopier proprement le texte suivant}

En astronomie, les nombres sont très grands. \\
La distance de la Terre au Soleil est \SI{152 000 000}{\km}. \\
L'étoile la plus proche (hormis le Soleil) est Proxima du Centaure. Elle se trouve à \SI{40 000 000 000 000}{km} de la Terre.\\
La masse de la Terre est \SI{5 980 000 000 000 000 000 000 000}{kg}.

\subsection*{ex2 - Recopier proprement le calcul suivant }

\begin{align*}
    B &= 3 \times 2 \times 5 + 10 - 6 \div 3  \\
    B &= 6 \times 5 + 10 - 2\\
    B &= 30 + 10 - 2 \\
    B &= 40 - 2\\
    B &= 38\\  
\end{align*}


\subsection*{ex1 - Recopier proprement le texte suivant}

En astronomie, les nombres sont très grands. \\
La distance de la Terre au Soleil est \SI{152 000 000}{\km}. \\
L'étoile la plus proche (hormis le Soleil) est Proxima du Centaure. Elle se trouve à \SI{40 000 000 000 000}{km} de la Terre.\\
La masse de la Terre est \SI{5 980 000 000 000 000 000 000 000}{kg}.

\subsection*{ex2 - Recopier proprement le calcul suivant }

\begin{align*}
    B &= 3 \times 2 \times 5 + 10 - 6 \div 3  \\
    B &= 6 \times 5 + 10 - 2\\
    B &= 30 + 10 - 2 \\
    B &= 40 - 2\\
    B &= 38\\  
\end{align*}


\subsection*{ex1 - Recopier proprement le texte suivant}

En astronomie, les nombres sont très grands. \\
La distance de la Terre au Soleil est \SI{152 000 000}{\km}. \\
L'étoile la plus proche (hormis le Soleil) est Proxima du Centaure. Elle se trouve à \SI{40 000 000 000 000}{km} de la Terre.\\
La masse de la Terre est \SI{5 980 000 000 000 000 000 000 000}{kg}.

\subsection*{ex2 - Recopier proprement le calcul suivant }

\begin{align*}
    B &= 3 \times 2 \times 5 + 10 - 6 \div 3  \\
    B &= 6 \times 5 + 10 - 2\\
    B &= 30 + 10 - 2 \\
    B &= 40 - 2\\
    B &= 38\\  
\end{align*}

\end{document}
