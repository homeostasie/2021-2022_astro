\documentclass[11pt]{article}
\usepackage{geometry,marginnote} % Pour passer au format A4
\geometry{hmargin=1cm, vmargin=1cm} % 

% Page et encodage
\usepackage[T1]{fontenc} % Use 8-bit encoding that has 256 glyphs
\usepackage[english,french]{babel} % Français et anglais
\usepackage[utf8]{inputenc} 

\usepackage{lmodern,numprint}
\setlength\parindent{0pt}

% Graphiques
\usepackage{graphicx,float,grffile}
\usepackage{pst-eucl, pst-plot,units} 

% Maths et divers
\usepackage{amsmath,amsfonts,amssymb,amsthm,verbatim}
\usepackage{multicol,enumitem,url,eurosym,gensymb}
\DeclareUnicodeCharacter{20AC}{\euro}

% Sections
\usepackage{sectsty} % Allows customizing section commands
\allsectionsfont{\centering \normalfont\scshape}

% Tête et pied de page

\usepackage{fancyhdr} \pagestyle{fancyplain} \fancyhead{} \fancyfoot{}

\renewcommand{\headrulewidth}{0pt} % Remove header underlines
\renewcommand{\footrulewidth}{0pt} % Remove footer underlines

\newcommand{\horrule}[1]{\rule{\linewidth}{#1}} % Create horizontal rule command with 1 argument of height

\usepackage{siunitx}
\sisetup{
    detect-all,
    output-decimal-marker={,},
    group-minimum-digits = 3,
    group-separator={~},
    number-unit-separator={~},
    inter-unit-product={~}
}

%----------------------------------------------------------------------------------------
%   Début du document
%----------------------------------------------------------------------------------------

\begin{document}

\setlength{\columnseprule}{1pt}

\subsection*{ex1 - Recopier proprement le texte suivant}

En astronomie, les nombres sont très grands. \\
La distance de la Terre au Soleil est \SI{152 000 000}{\km}. \\
L'étoile la plus proche (hormis le Soleil) est Proxima du Centaure. Elle se trouve à \SI{40 000 000 000 000}{km} de la Terre.\\
La masse de la Terre est \SI{5 980 000 000 000 000 000 000 000}{kg}.

\subsection*{ex2 - Recopier proprement le calcul suivant }

\begin{align*}
    B &= 3 \times 2 \times 5 + 10 - 6 \div 3  \\
    B &= 6 \times 5 + 10 - 2\\
    B &= 30 + 10 - 2 \\
    B &= 40 - 2\\
    B &= 38\\  
\end{align*}


\subsection*{ex1 - Recopier proprement le texte suivant}

En astronomie, les nombres sont très grands. \\
La distance de la Terre au Soleil est \SI{152 000 000}{\km}. \\
L'étoile la plus proche (hormis le Soleil) est Proxima du Centaure. Elle se trouve à \SI{40 000 000 000 000}{km} de la Terre.\\
La masse de la Terre est \SI{5 980 000 000 000 000 000 000 000}{kg}.

\subsection*{ex2 - Recopier proprement le calcul suivant }

\begin{align*}
    B &= 3 \times 2 \times 5 + 10 - 6 \div 3  \\
    B &= 6 \times 5 + 10 - 2\\
    B &= 30 + 10 - 2 \\
    B &= 40 - 2\\
    B &= 38\\  
\end{align*}


\subsection*{ex1 - Recopier proprement le texte suivant}

En astronomie, les nombres sont très grands. \\
La distance de la Terre au Soleil est \SI{152 000 000}{\km}. \\
L'étoile la plus proche (hormis le Soleil) est Proxima du Centaure. Elle se trouve à \SI{40 000 000 000 000}{km} de la Terre.\\
La masse de la Terre est \SI{5 980 000 000 000 000 000 000 000}{kg}.

\subsection*{ex2 - Recopier proprement le calcul suivant }

\begin{align*}
    B &= 3 \times 2 \times 5 + 10 - 6 \div 3  \\
    B &= 6 \times 5 + 10 - 2\\
    B &= 30 + 10 - 2 \\
    B &= 40 - 2\\
    B &= 38\\  
\end{align*}

\end{document}
