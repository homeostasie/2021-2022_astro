\documentclass[11pt]{article}
\usepackage{geometry,marginnote} % Pour passer au format A4
\geometry{hmargin=1cm, vmargin=1cm} % 

% Page et encodage
\usepackage[T1]{fontenc} % Use 8-bit encoding that has 256 glyphs
\usepackage[english,french]{babel} % Français et anglais
\usepackage[utf8]{inputenc} 

\usepackage{lmodern,numprint}
\setlength\parindent{0pt}

% Graphiques
\usepackage{graphicx,float,grffile,units}
\usepackage{tikz,pst-eucl,pst-plot,pstricks,pst-node,pstricks-add,pst-fun} 

% Maths et divers
\usepackage{amsmath,amsfonts,amssymb,amsthm,verbatim}
\usepackage{multicol,enumitem,url,eurosym,gensymb,tabularx}

\DeclareUnicodeCharacter{20AC}{\euro}



% Sections
\usepackage{sectsty} % Allows customizing section commands
\allsectionsfont{\centering \normalfont\scshape}

% Tête et pied de page
\usepackage{fancyhdr} \pagestyle{fancyplain} \fancyhead{} \fancyfoot{}

\renewcommand{\headrulewidth}{0pt} % Remove header underlines
\renewcommand{\footrulewidth}{0pt} % Remove footer underlines

\newcommand{\horrule}[1]{\rule{\linewidth}{#1}} % Create horizontal rule command with 1 argument of height

\newcommand{\Pointilles}[1][3]{%
  \multido{}{#1}{\makebox[\linewidth]{\dotfill}\\[\parskip]
}}

\newtheorem{Definition}{Définition}

\usepackage{siunitx}
\sisetup{
    detect-all,
    output-decimal-marker={,},
    group-minimum-digits = 3,
    group-separator={~},
    number-unit-separator={~},
    inter-unit-product={~}
}

\setlength{\columnseprule}{1pt}

\begin{document}

\textbf{Nom, Prénom :} \hspace{8cm} \textbf{Classe :} \hspace{3cm} \textbf{Date :}\\

\begin{center}
  \textit{Rien dans la vie n'est à craindre, tout doit être compris. C'est maintenant le moment de comprendre davantage, afin de craindre moins.}  - \textbf{Marie Curie}
\end{center}

\subsubsection*{1. Cours}

\textbf{Addition de fractions : } \dotfill \\
\Pointilles[1]

\subsubsection*{2. Démonstration}

\begin{multicols}{2}
\begin{enumerate}
  \item[1.] Démontrer que : $\dfrac{4}{9} + \dfrac{2}{9} = \dfrac{6}{9}$ 
  \item[2.] Démontrer que : $3 \times \dfrac{2}{19} = \dfrac{6}{19}$ 
\end{enumerate}
\end{multicols}

\Pointilles[14]

\subsubsection*{3. Calculer}
\textit{Calculer.}

\begin{multicols}{3}
\begin{enumerate}
  \item[a.] $\dfrac{20}{16} + \dfrac{14}{16}$ = \dotfill 
  \item[b.] $\dfrac{120}{39} + \dfrac{30}{39} - \dfrac{60}{39}$ = \dotfill 
  \item[c.] $\dfrac{4}{5} \times \dfrac{6}{3}$ = \dotfill 
\end{enumerate}
\end{multicols}

\subsubsection*{4. Fractions égales}
\textit{Écrire le nombre manquant.}

\begin{multicols}{3}
\begin{enumerate}
  \item[d.] $\dfrac{40}{30} = \dfrac{\ldots}{3}$
  \item[e.] $\dfrac{40}{10} = \dfrac{8}{\ldots}$
  \item[f.] $\dfrac{\ldots}{48} = \dfrac{7}{6}$
  \item[g.] $\dfrac{9}{\ldots} = \dfrac{72}{80}$
  \item[h.] $\dfrac{18}{6} = \dfrac{\ldots}{3}$
  \item[i.] $\dfrac{\ldots}{9} = \dfrac{16}{18}$ 
\end{enumerate}
\end{multicols}

\newpage
\subsubsection*{5. Calculer}
\textit{Calculer ; \textbf{Faire des étapes}.}

\begin{multicols}{5}
\begin{enumerate}
  \item[j.] $\dfrac{6}{4} \times \dfrac{3}{2} + \dfrac{7}{8}$
  \item[k.] $\dfrac{4}{5} + \dfrac{6}{15}$
  \item[l.] $4 + \dfrac{8}{3}$
  \item[m.] $\dfrac{1}{2} + \dfrac{1}{4} + \dfrac{1}{8}$
  \item[n.] $\dfrac{2}{3} + \dfrac{5}{3} \times \dfrac{5}{10}$
\end{enumerate}
\end{multicols}

\Pointilles[48]

\newpage


\textbf{Nom, Prénom :} \hspace{8cm} \textbf{Classe :} \hspace{3cm} \textbf{Date :}\\

\begin{center}
  \textit{Rien dans la vie n'est à craindre, tout doit être compris. C'est maintenant le moment de comprendre davantage, afin de craindre moins.}  - \textbf{Marie Curie}
\end{center}

\subsubsection*{1. Cours}

\textbf{Addition de fractions : } \dotfill \\
\Pointilles[1]

\subsubsection*{2. Démonstration}

\begin{multicols}{2}
\begin{enumerate}
  \item[1.] Démontrer que : $\dfrac{2}{27} + \dfrac{3}{27} = \dfrac{5}{27}$ 
  \item[2.] Démontrer que : $2 \times \dfrac{4}{13} = \dfrac{8}{13}$ 
\end{enumerate}
\end{multicols}

\Pointilles[14]

\subsubsection*{3. Calculer}
\textit{Calculer.}

\begin{multicols}{3}
\begin{enumerate}
  \item[a.] $\dfrac{40}{12} + \dfrac{6}{12}$ = \dotfill 
  \item[b.] $\dfrac{20}{14} + \dfrac{100}{14} - \dfrac{60}{14}$ = \dotfill 
  \item[c.] $\dfrac{6}{4} \times \dfrac{2}{5}$ = \dotfill 
\end{enumerate}
\end{multicols}

\subsubsection*{4. Fractions égales}
\textit{Écrire le nombre manquant.}

\begin{multicols}{3}
\begin{enumerate}
  \item[d.] $\dfrac{80}{50} = \dfrac{\ldots}{5}$
  \item[e.] $\dfrac{30}{10} = \dfrac{6}{\ldots}$
  \item[f.] $\dfrac{\ldots}{54} = \dfrac{8}{6}$
  \item[g.] $\dfrac{6}{\ldots} = \dfrac{42}{70}$
  \item[h.] $\dfrac{18}{6} = \dfrac{\ldots}{2}$
  \item[i.] $\dfrac{\ldots}{6} = \dfrac{16}{18}$ 
\end{enumerate}
\end{multicols}

\newpage
\subsubsection*{5. Calculer}
\textit{Calculer ; \textbf{Faire des étapes}.}

\begin{multicols}{5}
\begin{enumerate}
  \item[j.] $\dfrac{4}{6} \times \dfrac{3}{2} + \dfrac{11}{12}$
  \item[k.] $\dfrac{5}{8} + \dfrac{5}{24}$
  \item[l.] $5 + \dfrac{6}{4}$
  \item[m.] $\dfrac{1}{3} + \dfrac{1}{6} + \dfrac{1}{12}$
  \item[n.] $\dfrac{2}{4} + \dfrac{5}{4} \times \dfrac{5}{10}$
\end{enumerate}
\end{multicols}

\Pointilles[48]
\end{document}