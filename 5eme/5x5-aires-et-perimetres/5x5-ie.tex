\input{../doc-class-cours.tex}

\begin{document}

\textbf{Nom, Prénom :} \hspace{8cm} \textbf{Classe :} \hspace{3cm} \textbf{Date :}\\

\begin{center}
  \textit{Un tableau ne vit que par celui qui le regarde.}  - \textbf{Pablo Picasso}
\end{center}

\subsection*{ex1 : Restituer les formules de cours.}

\begin{multicols}{2}
\begin{itemize}
  \item Carré - Périmètre :\dotfill 
  \item Rectangle - Périmètre : \dotfill 
  \item Cercle - Périmètre : \dotfill
  \item Carré - Aire : \dotfill 
  \item Rectangle - Aire : \dotfill 
  \item Cercle - Aire : \dotfill
\end{itemize}
\end{multicols}

\subsection*{ex2 : Figures Composées}
Il faut calculer les aires et les périmètres des figures suivantes. 

\begin{multicols}{2}
  \begin{figure}[H]
    \centering
    \includegraphics[width=0.6\linewidth]{5x5-aires-et-perimetres/ex1.pdf}
  \end{figure}

  \begin{figure}[H]
    \centering
    \includegraphics[width=0.7\linewidth]{5x5-aires-et-perimetres/ex2.pdf}
  \end{figure}
\end{multicols}

\Pointilles[26]

\newpage


\begin{multicols}{2}
   \begin{figure}[H]
    \centering
    \includegraphics[width=0.65\linewidth]{5x5-aires-et-perimetres/ex3.pdf}
  \end{figure}
  
  \begin{figure}[H]
    \centering
    \includegraphics[width=0.7\linewidth]{5x5-aires-et-perimetres/ex4.pdf}
  \end{figure}
  
\end{multicols}

\Pointilles[37]




\end{document}