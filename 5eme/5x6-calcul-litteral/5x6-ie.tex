\input{../doc-class-cours.tex}

\begin{document}

\section*{Programme de l’éval : Calcul Littéral}

\textbf{Date : } \hspace{4cm} \textit{Calculatrice autorisée}

\subsection*{Cours - (2pts)}


\begin{itemize}[label={$\bullet$}]
  \item Sens du signe = : Ce qui est à gauche = Ce qui est à droite. 
  \item $x ?$ : Le nombre x est inconnu et solution de l’équation. 
\end{itemize}


\subsection*{Exercices Types}


\subsubsection*{Ex1. Calculer  - (3pts)}

On pose $x = 12$ et $b = 2,5$
\begin{itemize}[label={$\bullet$}]
  \item  $x + 2 - b = 11,5$
  \item  $5 \times x + 14 = 74$
  \item  $(2x - 4b) \times 4 = 136$
\end{itemize}

\subsubsection*{Ex2. Remplacer les $\times$ par des $+$ - (3pts)}

\begin{itemize}[label={$\bullet$}]
  \item  $2x + 3x = x+x + x+x+x$
  \item  $1 + 2x = 1 + x + x$        
\end{itemize}

\subsubsection*{Ex3. Réduire - (3pts)}

\begin{itemize}[label={$\bullet$}]
  \item  $2x + 3x = 5x$
  \item  $4x - x + 10 = 3x + 10$
\end{itemize}

\subsection*{Mise en équations - (9pts)}

\begin{itemize}[label={$\bullet$}]
  \item  Soit $x$ l’inconnu et la solution de mon problème.
  \item  Écrire l’égalité du problème.
\end{itemize}

Ex : Rania fait de la course à pied. Elle commence par courir 300m pour aller au stade. Elle fait 4 tours de terrain puis s’arrête. Son portable lui signale qu’elle a couru 3400m. Quelle est la longueur d’un tour de terrain ?

\begin{itemize}[label={$\bullet$}]
  \item  Soit $x$ la longueur d’un tour de terrain.
  \item  $300 + 4 \times x = 3400$
\end{itemize}

\end{document}