\input{../doc-class-cours.tex}

\begin{document}

\textbf{Nom, Prénom :} \hspace{8cm} \textbf{Classe :} \hspace{3cm} \textbf{Date :}\\

\begin{center}
  \textit{Les mathématiques, science de l’éternel et de l’immuable, sont la science de l’irréel.}  - \textbf{Ernest Renan}
\end{center}

\subsubsection*{Cours}

\begin{enumerate}
	\item[1.] Sens du signe = : \dotfill 
	\item[2.] Sens du x :  \dotfill  
\end{enumerate}

\subsubsection*{Ex1 : Calculer}

On pose $a = 15$ et $x = 0.2$.

\begin{enumerate}
  \item[a.] $a + 10$ = \dotfill 
  \item[b.] $a \times x + 25$ = \dotfill 
  \item[c.] $42 + 8 \times x$ = \dotfill 
  \item[d.] $(a + x)\times (a - x)$ = \dotfill 
  \item[e.] $2 \times x \times 5$ = \dotfill 
  \item[f.] $(2a + 10x) - 100x$ = \dotfill 
\end{enumerate}

\subsubsection*{Ex2 : Démontrer}

Écrire les calculs en représentant les multiplications avec des additions.

\begin{enumerate}
  \item[g.] $2 \times x + 4 \times x$ = \dotfill 
  \item[h.] $x + 3x + 4$ = \dotfill 
  \item[i.] $2x + 2 +  2x$ = \dotfill 
\end{enumerate}


\subsubsection*{Ex3 : Réduire}

\begin{enumerate}
  \item[k.] $12x + 20x$ = \dotfill 
  \item[l.] $4x + 6 + 40x$ = \dotfill 
  \item[m.] $8x - x$ = \dotfill 
  \item[n.] $4x - 3x + 12$ = \dotfill 
  \item[o.] $4 \times x \times 10$ = \dotfill 
  \item[p.] $7 \times x \times 0 + 13$ = \dotfill 
\end{enumerate}

\subsubsection*{Mise en équation}

\begin{enumerate}
  \item[1.] Rania va à Décathlon et achète 5 ballons de foot et 4 paires de gants de boxe. Chaque ballon coûte 12,99€. Le prix total afficher à la caisse est de 456,95€. Quel est le prix d'une paire de gants de boxe ?

  \item[2.] On a un rectangle avec une longueur six fois plus grande que sa largeur. L'aire du rectangle est $\SI{425}{cm^2}$. Quelle est sa largeur ?

  \item[3.] Mohammed sort de chez lui et court 450m avant d'arriver au stade. Il fait alors 6 tours de terrain en courant. Son portable lui indique qu'il a courut 2250m. Quelle la longueur d'un tour de stade ?
\end{enumerate}

\newpage

\textbf{Nom, Prénom :} \hspace{8cm} \textbf{Classe :} \hspace{3cm} \textbf{Date :}\\

\begin{center}
  \textit{Les mathématiques, science de l’éternel et de l’immuable, sont la science de l’irréel.}  - \textbf{Ernest Renan}
\end{center}

\subsubsection*{Cours}

\begin{enumerate}
	\item[1.] Sens du signe = : \dotfill 
	\item[2.] Sens du x :  \dotfill  
\end{enumerate}

\subsubsection*{Ex1 : Calculer}

On pose $a = 12$ et $x = 0.4$.

\begin{enumerate}
  \item[a.] $a + 10$ = \dotfill 
  \item[b.] $a \times x + 25$ = \dotfill 
  \item[c.] $42 + 8 \times x$ = \dotfill 
  \item[d.] $(a + x)\times (a - x)$ = \dotfill 
  \item[e.] $2 \times x \times 5$ = \dotfill 
  \item[f.] $(2a + 10x) - 100x$ = \dotfill 
\end{enumerate}

\subsubsection*{Ex2 : Démontrer}

Écrire les calculs en représentant les multiplications avec des additions.

\begin{enumerate}
  \item[g.] $3 \times x + 2 \times x$ = \dotfill 
  \item[h.] $4x + x + 4$ = \dotfill 
  \item[i.] $3x +3 + 3x$ = \dotfill 
\end{enumerate}


\subsubsection*{Ex3 : Réduire}

\begin{enumerate}
  \item[k.] $15x + 21x$ = \dotfill 
  \item[l.] $3x + 7 + 30x$ = \dotfill 
  \item[m.] $5x - x$ = \dotfill 
  \item[n.] $9x - 8x + 25$ = \dotfill 
  \item[o.] $3 \times x \times 10$ = \dotfill 
  \item[p.] $6 \times x \times 0 + 53$ = \dotfill 
\end{enumerate}

\subsubsection*{Mise en équation}

Écrire l'inconnu et modéliser le problème. 

\begin{enumerate}
  \item[1.] Rania va à Décathlon et achète 5 ballons de foot et 4 paires de gants de boxe. Chaque paires de gants de boxe coûte 15,99€. Le prix total afficher à la caisse est de 128,91€. Quel est le prix d'un ballon de foot ?

  \item[2.] On a un rectangle avec une longueur quatre fois plus grande que sa largeur. L'aire du rectangle est $\SI{425}{cm^2}$. Quelle est sa largeur ?

  \item[3.] Mohammed sort de chez lui et court 350m avant d'arriver au stade. Il fait alors 5 tours de terrain en courant. Son portable lui indique qu'il a courut 2450m. Quelle la longueur d'un tour de stade ?
\end{enumerate}

\end{document}