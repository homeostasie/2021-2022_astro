\input{../doc-class-cours.tex}

\begin{document}

\textbf{Nom, Prénom :} \hspace{8cm} \textbf{Classe :} \hspace{3cm} \textbf{Date :}\\

\begin{center}
  \textit{Une fausse bonne idée est de croire que la démocratie signifie que "mon ignorance est aussi bonne que votre connaissance.}  - \textbf{Isaac Asimov}
\end{center}

\subsection*{Cours : Tableau de proportionnalité :}

\Pointilles[3]

\subsection*{Exercices}

\textbf{EX1 - Calculer le coefficient de proportionnalité et remplir le tableau}\\

\begin{center}
  \begin{tabular}{|c|c|c|c|c|c|}
    \hline
    12 & 32                     & 54                     & 13                     & $\phantom{\dfrac{azertyuiop}{1}}$ & $\phantom{azertyuiop}$\\  \hline
    14 & $\phantom{\dfrac{azertyuiop}{1}}$ & $\phantom{azertyuiop}$ & $\phantom{azertyuiop}$ & 752                    &                    816\\  \hline
  \end{tabular}
\end{center}

\textbf{Coeff : }\dotfill \\

\textbf{EX2 - Calculs à trou}\\

\begin{multicols}{3}\noindent
\begin{itemize}[label={$\bullet$}]
  \item $5 \times \ldots \ldots = 3$ \\
  \item $14 \times \ldots \ldots = 23$ \\
  \item $15 \times \ldots \ldots = 32$ \\
  \item $51 \times \ldots \ldots = 16$ \\
  \item $504 \times \ldots \ldots = 3078$ \\
  \item $1768 \times \ldots \ldots = 703$ \\
\end{itemize}
\end{multicols}

\textbf{EX3 - Sens d'une fraction}\\

$\dfrac{4}{7}$ est le nombre qui \dotfill \\

\textbf{EX4 - Calcul exact et calcul approché}\\

\begin{itemize}[label={$\bullet$}]
  \item $5 \times \dfrac{9}{2} $ \dotfill \\
  \item $\dfrac{3}{7} \times 15 $  \dotfill \\
  \item $\dfrac{1}{2} \times \dfrac{27}{14} $  \dotfill \\
  \item $\pi \times 200 $  \dotfill \\
  \item $25 \times \dfrac{172}{204} $  \dotfill \\
\end{itemize}

\end{document}

